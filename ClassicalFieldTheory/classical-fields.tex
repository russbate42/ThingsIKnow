
%% This is a subsection of a chapter which can be imported

\section{Classical Field Theory Forumlation}
We begin with the Lagrangian
\begin{equation}
    \mathcal{L} = \mathcal{L}\left( \phi, \partial_\mu\phi \right)
\end{equation}
Note this lagrangian is a function of a scalar field $\phi$ and a space-time
derivative $\partial_\mu\phi$. This is a direct analogy to the classical lagrangian
\begin{equation}
    L = L\left( q, \dot{q} \right)
\end{equation}
Following the exact same procedure, we vary the Lagrangian $\mathcal{L}$.
\begin{align}
    \delta\mathcal{L} &= \pdv{L}{\phi}\delta\phi + \pdv{L}{\partial_\mu\phi}\delta(\partial_\mu\phi) \\
                      &= \pdv{L}{\phi}\delta\phi + \pdv{L}{\partial_\mu\phi}\partial_\mu(\delta\phi)
\end{align}
The next step involves a tricky substitution. This is often referred to as integration by parts,
but it is really just the chain rule. Both are true. Notice the following relationship.
\begin{equation}
    \partial_\mu\left( \pdv{\mathcal{L}}{\partial_\mu\phi} \delta\phi\right) = \left( \partial_\mu\pdv{\mathcal{L}}{\partial_\mu\phi} \right)\delta\phi + \pdv{\mathcal{L}}{\partial_\mu\phi} \partial_\mu(\delta\phi)
\end{equation}
Here I have placed brackets just to make the derivative explicit. This is precisely what we have in
the above equation. Thus, after the substitution this becomes

\begin{align}
    \delta\mathcal{L} &= \pdv{L}{\phi}\delta\phi + \partial_\mu\left( \pdv{\mathcal{L}}{\partial_\mu\phi} \delta\phi\right) - \left( \partial_\mu\pdv{\mathcal{L}}{\partial_\mu\phi} \right)\delta\phi \\
                      &= \left( \pdv{L}{\phi} - \partial_\mu\pdv{\mathcal{L}}{\partial_\mu\phi} \right)\delta\phi + \partial_\mu\left( \pdv{\mathcal{L}}{\partial_\mu\phi} \delta\phi\right)
\end{align}

Note that the last term when integrated is actually just a four divergence. This allows us to invoke
Stoke's theorem over all of space time. Analogously as in the mechanical case where the variation at the
boundaries was fixed at $(t_1,q_1)$, $(t_2,q_2)$, we impose the condition that the variation of the fields
reduces to zero at the boundary of space-time, taken to be infinity. Explicitly, $\delta\phi|_{R=\infty}=0$.

\begin{align}
    \int\mathcal{\delta L} &= \int\left( \pdv{L}{\phi} - \partial_\mu\pdv{\mathcal{L}}{\partial_\mu\phi} \right)\delta\phi\dv{x}^\mu + \int\partial_\mu\left( \pdv{\mathcal{L}}{\partial_\mu\phi} \delta\phi\right)\dd{x}^\mu \\
    0 &= \int\left( \pdv{L}{\phi} - \partial_\mu\pdv{\mathcal{L}}{\partial_\mu\phi} \right)\delta\phi\dd{x}^\mu + \partial_\mu\left( \pdv{\mathcal{L}}{\partial_\mu\phi} \delta\phi\right)\Bigg|_{R=\infty} \\
      &= \int\left( \pdv{L}{\phi} - \partial_\mu\pdv{\mathcal{L}}{\partial_\mu\phi} \right)\delta\phi\dd{x}^\mu
\end{align}

By construction, we vary the fields integrated over all of space-time. Therefore, it is not posisble for
the integration of all variations across space-time to equal zero. This implies that if this equation is
to be satisfied, the term in brackets must vanish. This is the \textit{Euler-Lagrange} equation of motion
for classical fields.
\begin{equation}
    \pdv{L}{\phi} - \partial_\mu\pdv{\mathcal{L}}{\partial_\mu\phi} = 0
\end{equation}

\subsection{Example - Klein-Gordon Equation}
Lemma
\begin{align}
    \pdv{\partial_\delta \phi} \left( \eta^{\mu\nu} \partial_\mu\phi \partial_\nu\phi \right) &= \eta^{\mu\nu}\left\{ \pdv{\partial_\nu\phi}{\partial_\delta\phi}\partial_\mu\phi + \partial_\nu\phi\pdv{\partial_\mu\phi}{\partial_\delta\phi} \right\} \\
     &= \eta^{\mu\nu}\left\{ \delta^\delta_\nu\partial_\mu\phi + \partial_\nu\phi\delta^\delta_\mu \right\} \\
     &= \left\{ \eta^{\mu\delta}\partial_\mu\phi + \eta^{\delta\nu}\partial_\nu\phi \right\} \\
     &= \left\{ \partial^\delta\phi + \partial^\delta\phi \right\}
\end{align}
Thus we can again use the space-time derivative with the $\delta$ dummy coordinate
\begin{align}
    \partial_\delta\pdv{\partial_\delta \phi} \left( \eta^{\mu\nu} \partial_\mu\phi \partial_\nu\phi \right) &= \partial_\delta\partial^\delta\phi + \partial_\delta\partial^\delta\phi \\
    &= 2\partial_\delta\partial^\delta\phi
\end{align}
In the future it will acceptable to write the following
\begin{align}
    \pdv{\partial_\mu\phi\partial^\phi}{\partial_\mu} &= \pdv{\partial_\mu}\left( \partial_\mu\phi \right)^2 \\
                                                      &= 2\partial_\mu\phi
\end{align}
This is potentially sloppy notation but is convenient short hand so it will be used.

\section{Symmetries With Classical Fields}

\section{Electricity and Magnetism}

\section{Particle Physics Kinematics}
\section{Summary}
\section{Cheat Sheet}
