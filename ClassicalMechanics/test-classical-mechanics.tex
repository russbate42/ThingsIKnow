\documentclass[12pt]{book}

\usepackage{import} % for importing alternative to include\input
\import{../}{preamble}
\import{../}{geometry}
\import{../}{header}
\usepackage{physics}

\begin{document}

\chapter{Introduction}
This is a test of the chapter on classical mechanics.

\chapter{Classical Mechanics}
\import{./}{classical-mechanics}
Begin with the statement about the action. The action can be described as
the difference between the kinetic and potential energy. $L = T - V$
\begin{equation}
    S = \int_T L(q,\dot{q}) \dv{t} 
\end{equation}
We then make the statement that the action is minimized
\begin{equation}
    \delta S = 0 \implies \delta \int_T L(q,\dot{q}) \dv{t} = 0
\end{equation}
Expand out the action
\begin{align}
    \delta L &= \pdv{L}{\dot{q}}\delta q + \pdv{L}{\dot{q}}\delta \dot{q} \\
             &= \pdv{L}{\dot{q}}\delta q + \pdv{L}{\dot{q}}\dv{t} q
\end{align}
Next we use a trick. This step is often referred to as ``integration by parts''
but it is really just the expension of a total derivative.
\begin{align}
    \dv{t}\left[ \pdv{L}{\dot{q}}\delta q \right] &= \left( \dv{t}\pdv{L}{\dot{q}} \right) \delta q + \pdv{L}{\dot{q}} \left( \dv{t} \delta q \right) \\
    \pdv{L}{\dot{q}} \left( \dv{t} \delta q \right) &= \dv{t}\left[ \pdv{L}{\dot{q}}\delta q \right] - \left( \dv{t}\pdv{L}{\dot{q}} \right)  
\end{align}
This enables the substitution.
\begin{align}
    \delta S &= 0 \\
    \int_T \delta L \dv{t} &= \int_T \left\{ \pdv{L}{q} \delta q + \dv{t}\left[ \pdv{L}{\dot{q}} \delta q \right] - \left( \dv{t}\pdv{L}{\dot{q}}\delta q \right) \right\} \dd t \\
                           &= \int_T \left[ \pdv{L}{q} - \dv{t}\pdv{L}{\dot{q}} \right]\delta q \dd t + \int_T \dv{t}\left[ \pdv{L}{\dot{q}}\delta q \right]\dd t \\
                           &= \int_T \left[ \pdv{L}{q} - \dv{t}\pdv{L}{\dot{q}} \right]\delta q \dd t + \pdv{L}{\dot{q}}\delta q \eval^{t_2}_{t_1}
\end{align}
With the last quantity, this depends on the boundary conditions. We assume that
the starting and ending points $q_1 and q_2$ are fixed at start and end times
$q_1$ and $q_2$ respectively. As the variation term $\delta q$ is zero here,
this entire term vanishes. Therefore, the only way that the action is zero is if
the quanitity in the square brackets vanishes.
\begin{equation}
    \implies \pdv{L}{q} - \dv{t}\pdv{L}{\dot{q}} = 0
\end{equation}
These are the \textbf{Euler-Lagrange equations of motion}.


\import{../}{bibliography.tex}

\end{document}

